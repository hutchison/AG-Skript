\subsection*{René Descartes}

\begin{itemize}
\item René Descartes (1596--1650) führte in seinem Werk \glqq La Géométrie\grqq\ Koordinaten in die
Geometrie ein
\item dadurch begründete er die \glqq Analytische Geometrie\grqq , die eine neue Verbindung von
Algebra und Geometrie ermöglichte
\item Geometrische Objekte ließen sich von nun an durch Gleichungen und Parametrisierungen
beschreiben
%\item Die Bedeutung dieser neuen Technik kann kaum überschätzt werden
\end{itemize}

Im Folgenden werden wir uns vor allem auf den Körper der reellen Zahlen beschränken. Sämtliche
Aussagen %???
haben ihre Gültigkeit aber auch für andere Körper. Wann immer von Ungleichungen die Rede ist, sollte
dabei ein geordneter Körper angenommen werden.

\subsection*{Kurven im $\R^2$}

Im $\R^2$ lassen sich {\em Kurven} beschreiben durch

\begin{enumerate}
\item[1)] Gleichungen

\begin{itemize}
\item lineare Gleichungen $a x + b y + c = 0$ mit $(a,b) \not = 0$ beschreiben Geraden
\item Kreise werden durch spezielle quadratischen Gleichungen $(x-a)^2 + (y-b)^2 - c^2 = 0$
beschrieben

%\item Gleichungen höheren Gerades sind auch von Interesse;
% Elliptische Kurven sind spezielle Kurven vom Grad $3$ die zum
% Beispiel in der Krytologie eine wichtige Rolle spielen

\end{itemize}

\item[2)] Parametrisierungen / Parameterdarstellungen

\begin{itemize}
\item Geraden werden parametrisiert mit Hilfe von Vektoren %einen Ortsvektor und einen Richtungsvektor
$$\left\{ \ve(xyz)\right\}$$
\item Kreise lassen sich mit Hilfe trigonometrischer Funktionen
  parametrisieren
$$
\left\{  \right\}
$$
\end{itemize}

\end{enumerate}

% ACHTUNG:
% Wir fassen hier den $\R^2$ als Vektorraum von Spaltenvektoren auf...


Die {\em algebraische Geometrie} befasst sich mit Darstellungen
geometrischer Objekte ({\em algebraischen Varietäten}) die als
Nullstellengebilde von Polynomgleichungen auftreten.


Die {\em Differentialgeometrie} befasst sich mit Parametrisierungen
geometrischer Objekte ({\em Mannigfaltigkeiten}) und daraus
abgeleiteten Eigenschaften.%, wie Krümmung


\medskip

{\bf To be filled: Ungleichungen bzw. zusätzliche Parameter 
ermöglichen die Beschreibung von Punkten im Inneren eines Kreises oder
auf einer Seite einer Geraden, etc...}





\subsection*{Lineare Algebra revisted}

Die Linearen Algebra befasst sich mit linearen Gleichungssystemen und
Parameterdarstellungen Ihrer Lösungen.

{\bf to be filled!}




\subsection*{Affine Geometrie}


\begin{defi}
Seien $x_1,\dots,x_m\in\R^n$ und
$\lambda_1,\dots,\lambda_m\in\R$. Dann heißt die Linearkombination
$x=\sum_{i=1}^m\lambda_i\,x_i $ genau dann 
{\em Affinkombination}, wenn $\sum_{i=1}^m\lambda_i = 1$.
\end{defi}



\begin{defi}
$x_1,\dots,x_m\in\R^n$ heißen
{\em affin (un)anhängig}, falls es
$(\lambda_1,\dots,\lambda_m)\in\R^m\setminus\{0\}$ gibt mit
$\sum_{i=1}^m\lambda_i = 0$ und $x=\sum_{i=1}^m\lambda_i\,x_i = 0$.
\end{defi}


\begin{thm}
$x_1,\dots,x_m\in\R^n$ sind genau dann affin unabhängig,
\begin{itemize}
\item
wenn sich keines der $x_i$ als Affinkombination der anderen $x_j$,
$j\not= i$, schreiben lässt.

\item
wenn $\begin{pmatrix}x_1\\1\end{pmatrix},\dots,\begin{pmatrix}x_m\\1\end{pmatrix}\in\R^{n+1}$ linear unabhängig sind
\end{itemize}
\end{thm}


%% ACHTUNG: Hier könnte man auch gut eine "affine Basis" einführen,
%% als maximal affin unabhängige Menge...




\section*{Vorlesung am 31.05.2011}







%%% Local Variables: 
%%% mode: latex
%%% TeX-master: "AxiomatischeGeometrie"
%%% End: 
