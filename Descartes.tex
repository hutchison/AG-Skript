\subsection*{René Descartes}

\begin{itemize}
\item René Descartes (1596--1650) führte in seinem Werk \glqq La Géométrie\grqq\ Koordinaten in die
Geometrie ein
\item dadurch begründete er die \glqq Analytische Geometrie\grqq , die eine neue Verbindung von
Algebra und Geometrie ermöglichte
\item Geometrische Objekte ließen sich von nun an durch Gleichungen und Parametrisierungen
beschreiben
%\item Die Bedeutung dieser neuen Technik kann kaum überschätzt werden
\end{itemize}

Im Folgenden werden wir uns vor allem auf den Körper der reellen Zahlen beschränken. Sämtliche
Aussagen %???
haben ihre Gültigkeit aber auch für andere Körper. Wann immer von Ungleichungen die Rede ist, sollte
dabei ein geordneter Körper angenommen werden.

\subsection*{Kurven im $\R^2$}

Im $\R^2$ lassen sich {\em Kurven} beschreiben durch

\begin{enumerate}
\item[1)] Gleichungen

\begin{itemize}
\item lineare Gleichungen der Form $a x + b y + c = 0$ mit $(a,b) \not = 0$ beschreiben Geraden
\item Kreise werden durch spezielle quadratischen Gleichungen $(x-a)^2 + (y-b)^2 - c^2 = 0$
beschrieben

%\item Gleichungen höheren Gerades sind auch von Interesse;
% Elliptische Kurven sind spezielle Kurven vom Grad $3$ die zum
% Beispiel in der Krytologie eine wichtige Rolle spielen

\end{itemize}

\item[2)] Parametrisierungen / Parameterdarstellungen

\begin{itemize}
\item Geraden werden parametrisiert mit Hilfe von Vektoren
%einen Ortsvektor und einen Richtungsvektor
$$
	\left\{
		(x,y) \in \R^2 \ | \ \ve{x}{y} = \ve{p_1}{p_2} + \lambda \ve{v_1}{v_2}, \lambda \in \R
	\right\}
$$

\item Kreise lassen sich mit Hilfe trigonometrischer Funktionen parametrisieren
$$
	\left\{
		(x,y) \in \R^2 \ | \ \ve{x}{y} = \ve{m_1}{m_2} + r \cdot \ve{\cos t}{\sin t}, t \in \R
	\right\}
$$
\hfill (Kreis mit Mittelpunkt $M = \ve{m_1}{m_2}$ und Radius $r$.)
\end{itemize}

\end{enumerate}

% ACHTUNG:
% Wir fassen hier den $\R^2$ als Vektorraum von Spaltenvektoren auf...

Die {\em algebraische Geometrie} befasst sich mit Darstellungen geometrischer Objekte ({\em
algebraischen Varietäten}) die als Nullstellengebilde von Polynomgleichungen auftreten.

Die {\em Differentialgeometrie} befasst sich mit Parametrisierungen geometrischer Objekte ({\em
Mannigfaltigkeiten}) und daraus abgeleiteten Eigenschaften.%, wie Krümmung

\medskip

{\bf To be filled: Ungleichungen bzw. zusätzliche Parameter ermöglichen die Beschreibung von Punkten
im Inneren eines Kreises oder auf einer Seite einer Geraden, etc...}

\subsection*{Lineare Algebra revisited}

Die Linearen Algebra befasst sich mit linearen Gleichungssystemen und Parameterdarstellungen Ihrer
Lösungen.

{\bf to be filled!}

\subsection*{Affine Geometrie}

\begin{bem}
	$\K^n, \R^n$: $\left\{ (x,y,z) \in \R^3 \ | \ ax + by + cz = d \right\}$ beschreibt eine 2-dim.
	Ebene im $\R^3$ (wenn $(a,b,c) \neq 0$)
\end{bem}

\begin{defi}
	Seien $x_1, \dots, x_m \in \R^n$ und $\lambda_1, \dots, \lambda_m \in \R$. Dann heißt die
	Linearkombination $x = \sum\limits_{i=1}^m \lambda_i \, x_i $ genau dann {\em Affinkombination},
	wenn $\sum\limits_{i=1}^m \lambda_i = 1$.
\end{defi}

\begin{defi}
	$x_1, \dots, x_m \in \R^n$ heißen {\em affin (un)anhängig}, falls es $(\lambda_1, \dots
	,\lambda_m) \in \R^m \setminus \{ 0 \}$ gibt mit $\sum\limits_{i=1}^m \lambda_i = 0$ und $x =
	\sum\limits_{i=1}^m \lambda_i \, x_i = 0$.
\end{defi}

\begin{thm}\label{def:affinabhaengig}
	$x_1, \dots, x_m \in \R^n$ sind genau dann affin unabhängig,
	\begin{itemize}
		\item wenn sich keines der $x_i$ als Affinkombination der anderen $x_j$, $j \neq i$, schreiben
		lässt.
		\item wenn $\ve{x_1}{1}, \dots, \ve{x_m}{1} \in \R^{n+1}$ linear unabhängig sind
	\end{itemize}
\end{thm}

%% ACHTUNG: Hier könnte man auch gut eine "affine Basis" einführen,
%% als maximal affin unabhängige Menge...

\section*{Vorlesung am 31.05.2011}

Wir betrachten folgendes Problem: wann liegen die Punkte
$$
	A = \ve{x_1}{y_1}, B = \ve{x_2}{y_2}, C = \ve{x_3}{y_3}
$$
auf einer Geraden (sind also {\em affin abhängig})?

Nach \ref{def:affinabhaengig} genau dann, wenn
$$
	\ved{x_1}{y_1}{1}, \ved{x_2}{y_2}{1}, \ved{x_3}{y_3}{1}
$$
linear abhängig sind. Dies ist genau dann der Fall, wenn
$$
	\Rang
	\underbrace{
	\begin{pmatrix}
		x_1 & x_2 & x_3\\
		y_1 & y_2 & y_3\\
		1 & 1 & 1
	\end{pmatrix}
	}_{=:M}
	\leq 2.
$$
Also genau dann, wenn ein Vektor $v = \ved{a}{b}{c} \in \R^3 \setminus \{ 0 \}$ mit $M^t \cdot v =
0$ existiert (also im Kern von $M^t$).

Betrachten wir das ausführlich, so ergibt sich:
$$
	\begin{pmatrix}
		x_1 & y_1 & 1\\
		x_2 & y_2 & 1\\
		x_3 & y_3 & 1
	\end{pmatrix}
	\cdot v = 0
	\quad \Leftrightarrow \quad
	\begin{matrix}
		ax_1 + by_1 + c = 0\\
		ax_2 + by_2 + c = 0\\
		ax_3 + by_3 + c = 0\\
	\end{matrix}
$$
Wir wissen: zwei Geraden mit den Gleichungen
\begin{align*}
	ax_1 + by_1 + c = 0 \quad \text{und} \quad ax_2 + by_2 + c = 0 & & \text{mit } (a_i,b_i) \neq
	(0,0)
\end{align*}
schneiden sich in einem Punkt $\ve{x}{y}$, wenn $\ved{x}{y}{1}$ die Lösung des homogenen
Gleichungssystems
\begin{equation}\label{eq:SchnittpunktzweierGeraden}
	\begin{pmatrix}
		a_1 & b_1 & c_1\\
		a_2 & b_2 & c_2
	\end{pmatrix}
	\ved{x}{y}{z}
	=
	\ve{0}{0}
\end{equation}
ist. Die beiden Geraden sind genau dann parallel, wenn $\ve{a_1}{b_1} = \lambda \ve{a_2}{b_2}$ für
ein $\lambda \in \R$ ist, d.h. wenn $\left\{ \mu \ved{-b_1}{a_1}{0} \ | \ \mu \in \R \right\}$
Lösungen von (\ref{eq:SchnittpunktzweierGeraden}) sind.

\subsection*{Strahlensatz revisited} % (fold)
\label{sub:Strahlensatz revisited}

\begin{thm}
	Sind $g(P,Q) \parallel g(B,C)$ und $Q = \mu C, P = \gamma B$ und $ABC$ ein Dreieck, so gilt $\mu
	= \gamma$.
	\begin{proof}
		\begin{align*}
			 & g(P,Q) \parallel g(B,C)\\
			\Leftrightarrow \ & P-Q = \lambda(B-C) & \text{für ein } \lambda \in \R\\
			\Rightarrow \ & P-Q = \gamma B - \mu C = \lambda B - \lambda C\\
			\Leftrightarrow \ & (\gamma - \lambda) B = (\mu - \lambda) C\\
			\Rightarrow \ & \gamma = \lambda = \mu, \text{weil $B,C$ linear abhängig sind}
		\end{align*}
	\end{proof}
\end{thm}

% subsection Strahlensatz revisited (end)

\subsection*{Konvexkombinationen} % (fold)
\label{sub:Konvexkombinationen}

\begin{defi}
	Seien $x_1, \dots, x_m \in \R^n$, dann heißt eine Affinkombination $x = \sum\limits_{i=1}^m
	\lambda_i \, x_i$ $\left( \text{mit } \sum\limits_{i=1}^m \lambda_i = 1 \right)$ {\em
	Konvexkombination} genau dann, wenn $\lambda_i \geq 0$ für $i = 1, \dots, m$.
\end{defi}

Im 2-dimensionalen müsste damit für die Konvexkombination $\lambda_1 x_1 + \lambda_2 x_2$ folgendes
gelten: $\lambda_1 +\lambda_2 = 1$ und $\lambda_1, \lambda_2 \geq 0$. Es gilt damit $\lambda_2 =
1-\lambda_1$ und der Mittelpunkt zwischen $x_1$ und $x_2$ ist $\frac{1}{2} x_1 + \frac{1}{2} x_2$.

\begin{thm}
	Seitenhalbierende schneiden sich in einem Punkt (dem Schwerpunkt $s = \frac{1}{3} x_1 +
	\frac{1}{3} x_2 + \frac{1}{3} x_3$).
	\begin{proof}
		Betrachte die Strecken $\lambda_1 (\frac{1}{2} x_1 + \frac{1}{2} x_2) + (1 - \lambda_1) x_3$
		und $\lambda_1 (\frac{1}{2} x_1 + \frac{1}{2} x_3) + (1 - \lambda_1) x_2$ mit $\lambda_1 \in
		[0,1]$. Wir erhalten den Schnittpunkt $s$ für $\lambda_1 = \frac{2}{3}$, was gezeigt werden
		musste.
	\end{proof}
\end{thm}

% subsection Konvexkombinationen (end)

\subsection*{$\R^n$ mit Standardskalarprodukt} % (fold)
%\label{sub:\R^n mit Standardskalarprodukt}

Das Standardskalarprodukt $\left\langle \cdot, \cdot \right\rangle: \R^n \times \R^n \to \R$ ist im
Euklidischen $n$-dimensionalen Vektorraum definiert durch
$$
	\left\langle x,y \right\rangle = x_1 y_1 + \dots + x_n y_n
$$

\begin{bem}
	Das Skalarprodukt bestimmt (bzw. ermöglicht) Längen- bzw. Winkelmessung durch
	\begin{align*}
		| x | & := \sqrt{\left\langle x,x \right\rangle}\\
		\alpha (x,y) & := \arccos \underbrace{\frac{\left\langle x,y \right\rangle}{| x | | y
		|}}_{\left\langle \frac{x}{| x |}, \frac{y}{|y|} \right\rangle}
	\end{align*}
\end{bem}

% subsection $\R^n$ mit Standardskalarprodukt (end)

\subsection*{Satz des \textsc{Pythagoras} revisited} % (fold)
\label{sub:Satz des Pythagoras revisited}

\begin{align*}
	| A - B |^2 & = \left\langle (A-B), (A-B) \right\rangle\\
	& = \underbrace{\left\langle A,A \right\rangle}_{|A|^2} + \underbrace{\left\langle B,B
	\right\rangle}_{|B|^2} - \underbrace{2 \left\langle A,B \right\rangle}_{2 |A| |B| \cdot \cos
	\theta}
\end{align*}
Nach dem Kosinussatz gilt dann $|A - B|^2 = |A|^2 + |B|^2 \Leftrightarrow A,B$ orthognal zueinander
sind ($\left\langle A,B \right\rangle = 0$).

% subsection Satz des Pythagoras revisited (end)

\subsection*{Satz des \textsc{Thales} revisited} % (fold)
\label{sub:Satz des Thales revisited}

Im Bild gilt:
\begin{align*}
	& \left\langle B-(-A), B-A \right\rangle = 0 & B \neq A\\
	\Leftrightarrow & \left\langle B,B \right\rangle - \left\langle A,A \right\rangle = 0\\
	\Leftrightarrow & |B|^2 = |A|^2
\end{align*}

\begin{center}
	\begin{tikzpicture}[line cap=round,line join=round,>=triangle 45,x=1.0cm,y=1.0cm]
	\coordinate [label=-45:$A$] (A) at (2,0);
	\fill (A) circle (2pt);
	\coordinate [label=-135:$-A$] (-A) at (-2,0);
	\fill (-A) circle (2pt);
	\coordinate [label=below:$O$] (O) at (0,0);
	\fill (O) circle (2pt);
	\draw (A) -- (-A);
	\begin{scope}
		\clip (-2.1,0) rectangle (2.1,2.1);
		\node [draw, circle, minimum size=4cm] (c) at (0,0) {};
	\end{scope}
	\coordinate [label=45:$B$] (B) at (intersection of (0,0) -- (2,3) and c);
	\fill (B) circle (2pt);
	\draw (-A) -- (B) -- (A);
	% Das muss gefixt werden. Geil ist das nicht.
	\draw ($ (B)!0.3!(A) $) let
			\p1 = ($ (B) - (A) $)
		in
			arc (-60:-151:{0.3*veclen(\x1,\y1)});
	\fill let
			\p1 = ($ (B)!0.03!(-A) $), \p2 = ($ (B)!0.17!(A) $)
		in
			(\x1,\y2) circle (1pt);
\end{tikzpicture}

\end{center}

% subsection Satz des Thales revisited (end)
