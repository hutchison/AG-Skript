% vim:tw=100:ts=4:sw=4:sts=4:et:

%Zitat von Arthur Cayley (1821-1895): "All geometry is projective geometry"
% siehe Wallner, Pottmann, Seite 1
\section{Vorlesung}

\subsection*{Gérard Desargues}

\begin{itemize}
    \item Hinweise zum Leben und seinen Werken

    \item Hinweise zum Leben und seinen Werken

    \item Hinweise zum Leben und seinen Werken
\end{itemize}

\subsection*{Costruzione Legittima}
Die ersten Anfänge der projektiven Geometrie liegen in der Malerei des $15.$ Jahrhunderts.  Zu
dieser Zeit entwickelte sich in Italien die Technik der \textsl{Costruzione Legittima}.  Mit dieser
Technik konnten erstmals räumlich erscheinende Bilder gezeichnet werden. Diese Technik ist jedoch
nicht nur für die Malerei bedeutend, sondern beinhaltet auch die Grundgedanken der projektiven
Geometrie. Dazu betrachten wir die folgenden Bilder.
\begin{figure}[ht]
    \begin{tikzpicture}[line cap=round,line join=round,>=triangle 45,x=1.0cm,y=1.0cm,xscale=1.5,scale=1.25]
  \draw [thick] (0,0) rectangle (6,4);	
  \draw [domain=0:6] plot(\x,{(--3-0*\x)/1});	
  \draw (3,3)-- (6,0);	
  \draw (3,3)-- (5,0);	
  \draw (3,3)-- (4,0);	
  \draw (3,3)-- (3,0);	
  \draw (3,3)-- (2,0);	
  \draw (3,3)-- (1,0);
  \draw (3,3)-- (0,0);
  \draw (1.84,1.84)-- (6,0);
  \draw [domain=0.63:5.37] plot(\x,{(--5.52-0*\x)/8.8});
  \draw [domain=1.04:4.96] plot(\x,{(--11.04-0*\x)/10.64});
  \draw [domain=1.33:4.67] plot(\x,{(--16.57-0*\x)/12.48});
  \draw [domain=1.54:4.46] plot(\x,{(--22.09-0*\x)/14.32});
  \draw [domain=1.71:4.29] plot(\x,{(--27.61-0*\x)/16.16});
  \draw [domain=1.84:4.16] plot(\x,{(--1.84-0*\x)/1});
	
\end{tikzpicture}
    \caption{Konstruktion mit Zirkel und Lineal}
\end{figure}
\begin{enumerate}[(a)] %\vspace{-0.5cm}
    \item Konstruiere Gerade als Horizont
    \item Konstruiere äquidistante Punkte am unteren Bildende
    \item Verbinde diese Punkte mit einem Punkt am Horizont
    \item Konstruiere Gerade "`quer"' durch das Bild, beginnend bei einem der äqudistanten Punkte
    \item Konstruiere Parallelen zum Horizont durch die entstandenen Schnittpunkte
\end{enumerate}

Intuitiv ist klar, dass die so erzeugten Rechtecke alle gleich groß sind, sie wirken nur
perspektivisch verkleinert. Ebeneso sind die Seiten der Rechtecke Parallelen, die sich jedoch am
Horizont schneiden. Damit liegt die Fordung nahe, dass sich Parallelen schneiden sollten, und zwar
in der Gerade unendlich ferner Punkte, dem Horizont.

Die Konstruktion unterliegt der Benutzung des Lineals und für die Parallelenkonstruktion des Zirkels
(\ref{Euklid-KonLotUndParallele}). Jedoch ist auch eine Konstruktion eines perspektivischen
gefließten Bodens nur mit dem Lineal möglich. Dies wird im zweiten Bild erläutert.

\begin{figure}[ht]
    \begin{tikzpicture}[line cap=round,line join=round,>=triangle 45,x=1.0cm,y=1.0cm,xscale=1.5,scale=1.25]

  \draw [thick] (0,0) rectangle (6,4);
  \draw [domain=0:6] plot(\x,{(--12-0*\x)/4});	
		
  \draw [domain=2.0:2.6163, color=black!40!white] plot(\x,{(-1.63331--0.8703*\x)/0.21458});
  \draw [color=black!40!white] (2.53249,0.94374)-- (2.61639,3);
  \draw [color=black!40!white] (1.75859,1.10351)-- (2.61639,3);
  \draw [color=black!40!white] (1.61108,1.47231)-- (2.61639,3);
  \draw [color=black!40!white] (2.90444,1.2537)-- (2.61639,3);
		
  \draw (1,3)-- (2,0.5);
  \draw (2,0.5)-- (5,3);
  \draw (2.53249,0.94374)-- (1,3);
  \draw (1.75859,1.10351)-- (5,3);

  \draw [domain=1.611:5.00] plot(\x,{(-2.23374-1.34971*\x)/-2.99409});
  \draw [domain=1.0:2.9044] plot(\x,{(--6.10299-1.42871*\x)/1.55809});
  % \draw [domain=1.0:3.1789] plot(\x,{(--6.75041-1.27186*\x)/1.82618}); 
  % irgendwie zeichnet er das nicht ordentlich, stattdessen
  \draw (1,3) -- (3.17895,1.48245); 
  \draw [domain=1.5115:5.0] plot(\x,{(-3.66563-1.15182*\x)/-3.14157});
	
\end{tikzpicture}
    \caption{Konstruktion mit Lineal}
\end{figure}
\begin{enumerate}[(a)]
    \item Konstruiere Gerade als Horizont
    \item Wähle $3$ Punkte auf Horizont einen weiteren unter dem Horizont
    \item Konstruiere Gerade durch die Punkte auf dem Horizont und dem frei gewählten
    \item wat soll das
    \item wat soll das
    \item wat soll das
\end{enumerate}

\subsection*{Axiomatisierung}

   Für eine Menge $\mathbb{P}$ von Punkten und Geraden (gewissen Teilmengen von
   $\mathbb{P}$) betrachten wir die \textit{Inzidenzaxiome}:

  \begin{enumerate}
    \item[{\bf(Ip1)}] Durch je zwei verschiedene Punkte gibt es genau eine Gerade.
    \item[{\bf(Ip2)}] Zwei verschiedene Geraden schneiden sich in genau einem Punkt.
    \item[{\bf(Ip3)}] Es gibt $4$ Punkte, von denen je $3$ nicht auf einer Geraden
                      liegen (nicht kollinear sind).
  \end{enumerate}


  \begin{defi}[projektive Ebene] \ \\
    Eine projektive Ebene ist eine Menge von Punkten und Geraden, die (\textbf{Ip1}),
    (\textbf{Ip2}) und (\textbf{Ip3}) erfüllen.

  \end{defi}

  \begin{description}
      \item[Modell 1] Endliche projektive Ebene \\
          Als Menge aller Punkte wählen wir $\mathbb{P} = \lbrace A,B,C,D,E,F,G \}$ mit dazugehörigen Geraden\\
          $\mathcal{G} = \big \{ \{ A,B,E \}, \{ C,D,E \}, \{ B,C,F \}, \{ A,D,F \},\{ A,C,G \}, \{ B,D,G \}, \{ E,F,G \} \big\}$ \\
          Wie man leicht nachprüfen kann, werden die $3$ Inzidenzaxiome dabei durch
          \textcolor{orange}{[nach (\textbf{Ip1}) \& (\textbf{Ip2})]},
          \textcolor{green}{[nach (\textbf{Ip1})]}, [nach (\textbf{Ip3})] erfüllt.
          Auch die sogenannte Fano-Ebene (nach Gino Fano\footnote{Gino Fano,
          $^*$ 5. Januar 1871 Mantua, $\dag$ 8. November 1952 Verona, italienischer Mathematiker})
          erfüllt die Inzidenzaxiome (Ip1), (Ip2) und (Ip3). In der Fano-Ebene wird der Kreis
          durch die Punkte $B,C$ und $F$ als Gerade aufgefasst. Nur so können die Inzidenzaxiome
          erfüllt werden. Im Laufe des Kapitels werden wir noch öfter auf dieses Modell verweisen.

          \begin{figure}[ht]
            \begin{tabular}{cc}
              \begin{tikzpicture}

  \clip (-0.5,-0.5) rectangle (4.5,3.6);
			
  \draw (0,0) coordinate (A);
  \draw (0.75,1.5) coordinate (B);
  \draw (2.25,1.5) coordinate (C);
  \draw (3,0) coordinate (D);
  \draw (1.5,3) coordinate (E);
  \draw (4,1) coordinate (F);
  \draw (1.5,1) coordinate (G);
			
  \draw (A)--(B)--(C)--(D)--(A);
  \draw [color=orange] (A)--(C)--(E)--(B)--(D)--(F)--(C);
  \draw [color=green]  (E)--(G)--(F);
			
  \fill [color=colPkt] (A) circle (1.5pt) node[below left]  {\footnotesize $A$};
  \fill [color=colPkt] (B) circle (1.5pt) node[above left]  {\footnotesize $B$};
  \fill [color=colPkt] (C) circle (1.5pt) node[above right] {\footnotesize $C$};
  \fill [color=colPkt] (D) circle (1.5pt) node[below right] {\footnotesize $D$};
  \fill [color=colPkt] (E) circle (1.5pt) node[above] {\footnotesize $E$};
  \fill [color=colPkt] (F) circle (1.5pt) node[right] {\footnotesize $F$};
  \fill [color=colPkt] (G) circle (1.5pt) node[below] {\footnotesize $G$};
		
\end{tikzpicture}
              &
              \begin{tikzpicture}[scale=0.5]
			
  \clip (-1,-1) rectangle (7,6);
			
  \draw (0,0)       coordinate (A);		
  \draw (1.5,2.598) coordinate (B);		
  \draw (4.5,2.598) coordinate (C);		
  \draw (6,0)       coordinate (D);		
  \draw (3,5.196)   coordinate (E);		
  \draw (4.5,2.598) coordinate (C);		
  \draw (3,0)       coordinate (F);		
  \draw (3,1.732)   coordinate (G);
					
  \draw (A)--(E)--(D)--(A);		
  \draw (A)--(C);		
  \draw (B)--(D);		
  \draw (E)--(F);	
			
  \draw (G) circle (1.73205cm);
					
  \fill [color=colPkt] (A) circle (3pt) node[below left]  {\footnotesize $A$};	
  \fill [color=colPkt] (B) circle (3pt) node[left]  {\footnotesize $B$};	
  \fill [color=colPkt] (C) circle (3pt) node[right] {\footnotesize $C$};	
  \fill [color=colPkt] (D) circle (3pt) node[below right] {\footnotesize $D$};	
  \fill [color=colPkt] (E) circle (3pt) node[above] {\footnotesize $E$};	
  \fill [color=colPkt] (F) circle (3pt) node[below] {\footnotesize $F$};	
  \fill [color=colPkt] (G) circle (3pt) node[below=2pt, left=5pt] {\footnotesize $G$};
		
\end{tikzpicture}
            \end{tabular}
            \caption{endliche projektive Ebene und Fano-Ebene}
          \end{figure}

      \item[Modell 2] Reelle projektive Ebene $\Pro_2(\R)$ \label{Desargues-ReelleProjektiveEbene} \\
          In der reellen projektiven Ebene $\mathbb{P}_2(\R)$ wählen wir
          $\mathbb{P} = \{ \mbox{Geraden im $\R^3$ durch $0$} \}$ und analog
          $\mathcal{G} = \{ \mbox{Ebenen im $\R^3$ durch $0$} \}$. In gewisser Weise werden dadurch
          Punkte zu Geraden und Geraden zu Ebenen. Daher werden Punkte in diesem Modell auch als
          (projektive) Punkte und Geraden als (projektive) Geraden bezeichnet.
          \begin{align*}
              &&\text{(projektiver) Punkte:} &
                  \langle v \rangle = \{ \lambda \cdot v \mid \lambda \in \R \} \text{ für ein } v\in \R^3 \backslash \{ 0 \} \\
              &&\text{(projektive) Gerade:} &
                  \Big\{ \begin{pmatrix} x \\ y \\ z\end{pmatrix} \in \R^3
                  \mid ax + by + cz = 0 \wedge a,b,c \in \R\backslash \{0\} \Big\} =
                  \begin{pmatrix} a \\ b \\ c\end{pmatrix}^\perp
          \end{align*}
          Somit lässt sich die projektive Gerade vollständig über das orthogonale Komplement
          von $\begin{pmatrix} a & b & c\end{pmatrix}^T$, bzw. durch den projektiven Punkt
          $\langle \begin{pmatrix} a & b & c\end{pmatrix}^T \rangle$ beschreiben.
  \end{description}



  \begin{bem}[Die reelle projektive Ebene $\mathbb{P}_2(\R)$ erfüllt die Inzidenzaxiome.] \ \\
    (\textbf{Ip1}) $\&$ (\textbf{Ip2}) sind bekannte Regeln aus der linearen Algebra. Das dritte
    Inzidenzaxiom (\textbf{Ip3}) gilt zum Beispiel für die projektiven Punkte
    \begin{align*}
      \langle \begin{pmatrix} 1\\0\\0 \end{pmatrix} \rangle,
      \langle \begin{pmatrix} 0\\1\\0 \end{pmatrix} \rangle,
      \langle \begin{pmatrix} 0\\0\\1 \end{pmatrix} \rangle,
      \langle \begin{pmatrix} 1\\1\\1 \end{pmatrix} \rangle
      \intertext{Für die projektive Gerade durch die ersten beiden Punkte gilt beispielsweise}
      g\left( \langle \begin{pmatrix} 1\\0\\0 \end{pmatrix} \rangle, \langle \begin{pmatrix} 0\\1\\0 \end{pmatrix} \rangle \right)
          = \left \{ \begin{pmatrix} x\\y\\z \end{pmatrix} \in \R^3 \mid z = 0 \right \}
    \end{align*}
    Diese Ebene (projektive Gerade) enthält jedoch nicht die beiden anderen Punkte. Analog prüft man dies für die anderen Punktepaare.
  \end{bem}

\subsection*{Rechnen mit homogenen Koordinaten}
  Nach der Definition in Modell 2 \ref{Desargues-ReelleProjektiveEbene} entsprechen Punkte in $\mathbb{P}_2(\R)$ Geraden
  und Geraden in $\mathbb{P}_2(\R)$ Ebenen durch den Ursprung unseres Koordinatensystems. Daher ist es sinnvoll Punkte
  nicht einfach nur als $3$-Tupel $(x,y,z)$ zu betrachten. Stattdessen betrachtet man die homogenen Koordinaten des Punktes.

  \begin{defi}[homogene Koordinaten] \ \\
    Die homogenen Koordinaten eines projektiven Punktes $P$ in der reellen projektiven Ebene $\mathbb{P}_2(\R)$ sind alle
    $v\in \R^3\backslash\{0\}$ mit $P=\langle v \rangle$.
  \end{defi}

  Damit ergeben sich auch andere Vorstellung bzgl. des Schnittes zweier projektiven Punkte, oder zweier projektiven Ebene.
  Durch die Geraden- oder Ebenedarstellung der projektiven Puntke, Geraden können wir für die Schnitte die Mittel der Linearen
  Algebra nutzen. \par \medskip

  \textbf{Berechnung der Verbindungsgerade zweier Punkte} \par
  Gegeben sind die projektiven Punkte (Geraden) $P=\langle v \rangle$ und $Q=\langle w \rangle$
  mit $P\neq Q$. Die projektive Verbingungsgerade zwischen diesen beiden Punkten ist eine Ebene,
  die sich über den Normalenvektor ausdrücken lässt. So ergibt sich die Verbindungsstrecke als:
  \par
  \begin{center}
    $g(P,Q) = P \vee Q = \big( v \times w \big)^\perp$
  \end{center} \par
  mit der Abbildung "`Join"' $\vee: \mathbb{P} \times \mathbb{P} \rightarrow \mathcal{G}$. In
  der Skizze ist die Ebene angedeutet.
  \begin{figure}[ht]
    \tdplotsetmaincoords{70}{115}		% ---Betrachtung auf das Koordinatensystem---
\begin{tikzpicture}[line cap=round, line join=round, tdplot_main_coords,>=triangle 45,scale=2]

  \filldraw [color=yellow!90!white, semitransparent] (0,0,0) -- (1,1.5,0) -- (-1,1.5,0) --cycle;
  \draw (0.2,1.15,0) node{\footnotesize $\big(v \times w \big)^\perp$};

  \draw [-stealth,color=black] (0,0,0) -- (-1,1.5,0) node[right]{\footnotesize $w$};
  \draw [-stealth,color=black] (0,0,0) -- (1,1.5,0)  node[right]{\footnotesize $v$};
  \draw [-stealth,color=black] (0,0,0) -- (0,0,1)    node[above]{\footnotesize $v\times w$};

\end{tikzpicture}
    \caption{projektive Verbindungsgerade in der reellen projektiven Ebene}
  \end{figure}

  \textbf{Berechnung von Schnittpunkten zweier Geraden} \par
  Gegeben seien die projektiven Geraden $g=v^\perp$ und $h=w^\perp$ mit $g\neq h$. Für den
  projektiven Schnittpunkt beider projektiven Geraden erhält man einen projektiven Punkt.
  Dieser ergibt sich zu:
  \begin{center}
    $P = g \wedge h = v \times w $
  \end{center}
  mit der Abbildung "`Meet"' $\wedge: \mathcal{G} \times \mathcal{G} \rightarrow \mathbb{P}$
  \begin{figure}[ht]
    \tdplotsetmaincoords{70}{110}		% ---Betrachtung auf das Koordinatensystem---
\begin{tikzpicture}[line cap=round, line join=round, tdplot_main_coords,>=triangle 45,scale=0.75]
			
  \filldraw [color=yellow!90!white, semitransparent] (1,-3,-2) -- (-3,1,-2) -- (-1,3,2) --  (3,-1,2) -- cycle;					
  \filldraw [color=yellow!90!white, semitransparent] (1,-3,2) -- (-3,1,2) -- (-1,3,-2) --  (3,-1,-2) -- cycle;
	
  \draw[-stealth] (-2,2,0)--(2,-2,0) node[left] {$v \times w$};	
  \draw[-stealth] (0,0,0)--(-1,-1,2) node[above] {$v$};
  \draw[-stealth] (0,0,0)-- (1,1,2)  node[above] {$w$};

\end{tikzpicture}
    \caption{Schnittpunkt zweier projektiven Geraden}
  \end{figure}


\subsection*{Dualitätsprinzip}

  Zu jeder projektiven Ebene $\big( \mathbb{P}, \mathcal{G}, \vee, \wedge \big)$ gibt es eine duale projektive Ebene, in
  der die Rollen von $\mathbb{P}$ und $\mathcal{G}$ und von $\vee$ und $\wedge$ vertauscht sind.
  $\big( \mathcal{G}, \mathbb{P}, \wedge, \vee \big) - \big( \mathbb{P}^*$, $ \mathcal{G}^*, \vee, \wedge \big)$

  Der Grundgedanke dieses Dualitätsprinzips ist, dass in einer wahren Aussagen die Rollen von Punkt und Gerade, sowie von
  Verbindungsgerade zwischen zwei Punkten und Schnittpunkt zweier Geraden vertauscht werden können, ohne dass die wahre
  Aussage ihre Gültigkeit verliert.

  \begin{bem} Insbesondere gelten
    \begin{enumerate}
      \item $\mathbb{P}_2(\R)$ ist selbstdual
      \item Die Fano-Ebene ist selbstdual
      \item (Ip1)$^*$ = (Ip2) und (Ip2)$^*$=(Ip1)
    \end{enumerate}
  \end{bem}

  \begin{thm}[Dualität von (\textbf{Ip3}) - (\textbf{Ip3})$^*$] \ \\
    In einer projektiven Ebene gibt es vier Geraden, von denen je $3$ keinen Punkt gemein haben.
  \end{thm}

  \begin{description}
      \item[Modell 3] projektive Ebene $\mathbb{P}_2(\K)$ über einem Körper $\K$ \\
          Auch hier werden die projektiven Punkte wie in Modell 2, diesemal jedoch über einem beliebigen Körper $\K$, definiert.
          Die projektive Ebene ergibt sich dann durch
          \begin{align*}
            \mathbb{P}_2(\K) = \Big( \K^3\backslash \{0\}\Big) \Big/_\sim \quad
            \mbox{wobei} \quad v \sim w : \quad \Leftrightarrow \quad v=\lambda w
            \mbox{ für } \lambda \in \K
          \end{align*}
          erzeugt. $v,w \in \K^3\backslash\{0\}$ heißen homogene Koordinaten der Punkte $P=\langle v \rangle$, $Q=\langle w \rangle$.
  \end{description}

  \begin{bsp}
    Für $\K=\K_2 = \{0,1\}$, den Körper mit zwei Elementen erhält man
    \begin{align*}
      \K_2^3 \backslash \{0\} \Big/_\sim = \mathbb{P}_2(\K_2)
          & = \left\{ \langle \begin{pmatrix} 1 \\ 0 \\ 0 \end{pmatrix} \rangle,
                      \langle \begin{pmatrix} 0 \\ 1 \\ 0 \end{pmatrix} \rangle,
                      \langle \begin{pmatrix} 0 \\ 0 \\ 1 \end{pmatrix} \rangle,
                      \langle \begin{pmatrix} 1 \\ 1 \\ 0 \end{pmatrix} \rangle,
                      \langle \begin{pmatrix} 1 \\ 0 \\ 1 \end{pmatrix} \rangle,
                      \langle \begin{pmatrix} 0 \\ 1 \\ 1 \end{pmatrix} \rangle,
                      \langle \begin{pmatrix} 1 \\ 1 \\ 1 \end{pmatrix} \rangle
              \right\}
    \end{align*}
    als die homogenen Koordinaten der Fano-Ebene.
  \end{bsp}

\subsection*{projektive Automorphismen}

  \begin{defi}
    Jede reguläre Abbildung/Matrix $M\in$ GL$_3(\K)$ induziert eine projektive Abbildung
    \begin{align*}
      \mathbb{P}_2(\K) \rightarrow  \mathbb{P}_2(\K) \quad \mbox{durch} \quad
      \langle x \rangle \mapsto \langle Mx \rangle \qquad x \in \K^{3}\backslash\{0\}
    \end{align*}
  \end{defi}

  \begin{thm} Die projektiven Abbildungen bilden eine Gruppe.
    \begin{align*}
      \mathbb{P} \mbox{GL}_2(\K) = \mbox{GL}_3(\K) \Big/ \big( \R \cdot \Id \big)
    \end{align*}
  \end{thm}

  \begin{proof}
    Der fehlt mir...
  \end{proof}

  SPHÄRISCHES MODELL

  \begin{thm} [projektive Standardbasis] \ \\
    Zu einem vollständigen Viereck ABCD (je $3$ Punkte sind nicht kollinear) mit $A=\langle a \rangle$, $B=\langle b \rangle$,
    $C=\langle c \rangle$ und $D=\langle d \rangle \in \mathbb{P}_2(\K)$ gibt es eine projektive Abbildung
    $\mathcal P: \mathbb{P}_2(\K) \rightarrow \mathbb{P}_2(\K)$ auf die projektive Standardbasis
    \begin{align*}
      \mathcal{P}(A) = \langle \begin{pmatrix} 1 \\ 0 \\ 0 \end{pmatrix} \rangle \quad
      \mathcal{P}(B) = \langle \begin{pmatrix} 0 \\ 1 \\ 0 \end{pmatrix} \rangle \quad
      \mathcal{P}(C) = \langle \begin{pmatrix} 0 \\ 0 \\ 1 \end{pmatrix} \rangle \quad
      \mathcal{P}(D) = \langle \begin{pmatrix} 1 \\ 1 \\ 1 \end{pmatrix} \rangle ,
    \end{align*}
    bzw. es gibt eine Matrix $M\in \mbox{GL}_3(\K)$ mit
    \begin{align*}
      \langle M \cdot a \rangle = \langle \begin{pmatrix} \lambda \\ 0 \\ 0 \end{pmatrix} \rangle \quad
      \langle M \cdot b \rangle = \langle \begin{pmatrix} 0 \\ \lambda \\ 0 \end{pmatrix} \rangle \quad
      \langle M \cdot c \rangle = \langle \begin{pmatrix} 0 \\ 0 \\ \lambda \end{pmatrix} \rangle \quad
      \langle M \cdot d \rangle = \langle \begin{pmatrix} \lambda \\ \lambda \\ \lambda  \end{pmatrix} \rangle, \quad \lambda \in \K
    \end{align*}
  \end{thm}

  \begin{proof}
    Den hab ich zwar (beim mir is es ein 2-zeiler), verstehe ihn aber nicht... :D
  \end{proof}

  \begin{thm}[Theorem von Pappos\footnote{Pappos von Alexandria, ca. $300$ n. Chr., griechischer Mathematiker}] \ \\
    In $\mathbb{P}_2(\K)$ seien zwei verschiedene Geraden $g$ und $g'$, sowie paarweise verschiedene Punkte $A,B,C \in g$ und $A',B',C' \in g'$ von denen keiner
    $S=g\cap g'$ ist. Dann liegen die Punkte
    \begin{align*}
      (A\vee B') \wedge (A'\vee B) := P \quad (B\vee C') \wedge (B'\vee C) := Q \quad (A\vee C') \wedge (A'\vee C) := R
    \end{align*}
    auf einer gemeinsamen Geraden.

    \begin{figure}[ht]
      \definecolor{ccqqcc}{rgb}{0.8,0,0.8}		% lila
\definecolor{ffzzqq}{rgb}{1,0.6,0}			% orange	
\definecolor{qqccqq}{rgb}{0,0.8,0}			% gr�n
\definecolor{xdxdff}{rgb}{0.49,0.49,1}	
\definecolor{qqqqff}{rgb}{0,0,1}		
\begin{tikzpicture}[line cap=round,line join=round,>=triangle 45,x=1.0cm,y=1.0cm]
				
  \clip(-0.5,-0.5) rectangle (8.5,3);
  \draw (0,0)       coordinate (S);
  \draw (1.5,0.75)  coordinate (A);
  \draw (2.75,1.37) coordinate (B);
  \draw (4.57,2.29) coordinate (C);
  \draw (1.93,0)    coordinate (A');
  \draw (4.57,0)    coordinate (B');
  \draw (7.63,0)    coordinate (C');
  \draw (2.2654,0.5628) coordinate (P);
  \draw (2.6363,0.6109) coordinate (Q);
  \draw (4.5704,0.8612) coordinate (R);

  \draw [color=qqccqq] (A)--(B');
  \draw [color=qqccqq] (B)--(A');
  \draw [color=ccqqcc] (A)--(C');
	\draw [color=ccqqcc] (C)--(A');
  \draw [color=ffzzqq] (B)--(C');
  \draw [color=ffzzqq] (C)--(B');

  \draw [domain=-0.5:8.5] plot(\x,{(-0--0.75*\x)/1.5});
  \draw [domain=-0.5:8.5] plot(\x,{(-0-0*\x)/1.93});
  \draw [dash pattern=on 4pt off 4pt,domain=-0.67155:7.83438] plot(\x,{(--0.1--0.04802*\x)/0.37095});
				
				
  \fill [color=colPkt] (S)  circle (1.5pt) node[below] {\footnotesize $S$};
  \fill [color=colPkt] (A)  circle (1.5pt) node[above] {\footnotesize $A$};
  \fill [color=colPkt] (B)  circle (1.5pt) node[above] {\footnotesize $B$};
  \fill [color=colPkt] (C)  circle (1.5pt) node[above] {\footnotesize $C$};
  \fill [color=colPkt] (A') circle (1.5pt) node[below] {\footnotesize $A'$};
  \fill [color=colPkt] (B') circle (1.5pt) node[below] {\footnotesize $B'$};
  \fill [color=colPkt] (C') circle (1.5pt) node[below] {\footnotesize $C'$};
  \fill [color=qqccqq] (P)  circle (1.5pt) node[above] {\footnotesize $P$};
  \fill [color=ccqqcc] (Q)  circle (1.5pt) node[above] {\footnotesize $Q$};
  \fill [color=ffzzqq] (R)  circle (1.5pt) node[above right] {\footnotesize $R$};
  \draw (5.9,2.9) node[below] {\footnotesize $g$};
  \draw (8.2,0)   node[above] {\footnotesize $g'$};
\end{tikzpicture}
      \caption{Satz von Pappos}
    \end{figure}

  \end{thm}

  \begin{proof}
    Bilde $S, A, A'$ und $P$ projektiv auf die projektive Standardbasis ab. Festlegung:
    \begin{align*}
      S  =  \begin{pmatrix} 1 \\ 0 \\ 0 \end{pmatrix} \qquad
      A  =  \begin{pmatrix} 0 \\ 1 \\ 0 \end{pmatrix} \qquad
      A' =  \begin{pmatrix} 0 \\ 0 \\ 1 \end{pmatrix} \qquad
      P  =  \begin{pmatrix} 1 \\ 1 \\ 1 \end{pmatrix}
    \end{align*}
    Daraus ergeben sich dann die homogenen Koordinaten für die restlichen Punkte. So liegt zum Beispiel der Punkt $B$ auf
    der Geraden $S \vee A$. $B$ ergibt sich demnach als Linearkombination der Punkte $A$ und $P$. Da es sich um homogene
    Koordinaten handelt, ist $B=\begin{pmatrix} 1 & 1 & 0 \end{pmatrix}^T$. Durch analoge Überlegungen erhält man für die
    restlichen Punkte $A,B,C,A',B',C'$
    \begin{align*}
      \begin{pmatrix} 0 \\ 1 \\ 0 \end{pmatrix},
      \begin{pmatrix} 1 \\ 1 \\ 0 \end{pmatrix},
      \begin{pmatrix} \gamma  \\ 1 \\ 0 \end{pmatrix},
      \begin{pmatrix} 0 \\ 0 \\ 1 \end{pmatrix},
      \begin{pmatrix} 1 \\ 0 \\ 1 \end{pmatrix},
      \begin{pmatrix} \gamma' \\ 0 \\ 1 \end{pmatrix}
    \end{align*}

    Die homogenen Koordinaten der Punkte $P,Q$ und $R$ sind demnach
    \begin{align*}
      \begin{pmatrix} 1 \\ 1 \\ 1 \end{pmatrix} \qquad
      \begin{pmatrix} \gamma\gamma'-1\\ \gamma'-1\\ \gamma-1 \end{pmatrix} \qquad
      \begin{pmatrix} \gamma\gamma'  \\ \gamma'  \\ \gamma   \end{pmatrix} \qquad
    \end{align*}
    Diese liegen alle in einer Ebene, bzw. einer projektiven Geraden.
  \end{proof}

  Man kann zeigen, dass der Satz von Pappos den Satz von Desargues impliziert (Hessenberg $1905$).

  \begin{thm}[Theorem von Desargues] \label{Theorem-von-Desargues}\ \\
    Wenn zwei Dreiecke in Perspektive eines Punktes liegen, so liegen sie auch in Perspektive bzgl. einer Geraden. In Perspektive
    eines Punktes bedeutet dabei: Es gibt einen Punkt $P\in \mathbb{P}_2(\K)$, sodass für die Dreiecke $ABC$ und $A'B'C'$ gilt:
    $P\in \big\{ (A\vee A'), (B\vee B'), (C\vee C') \big\}$.

    \begin{figure}[ht]
      \begin{tikzpicture}[line cap=round,line join=round,>=triangle 45,x=1.0cm,y=0.8cm]
				
  \clip(-0.5,-1.5) rectangle (6.5,2.5);
  \draw (0,0) 	 coordinate (P);	
  \draw (2,1) 	 coordinate (A);	
  \draw (1.5,0)  coordinate (B);	
  \draw (2,-0.5) coordinate (C);
  \draw (4,2)    coordinate (A');	
  \draw (5,0)    coordinate (B');	
  \draw (4,-1)   coordinate (C');
	
  \draw [domain=0.0:9.5, color=black!50!white] plot(\x,{(-0--1*\x)/2});
  \draw [domain=0.0:9.5, color=black!50!white] plot(\x,{(-0-0*\x)/3});
  \draw [domain=0.0:9.5, color=black!50!white] plot(\x,{(-0-0.5*\x)/2});
				
  \draw (A)--(B)--(C)--cycle;
  \draw (A')--(B')--(C')--cycle;
				
  \fill [color=colPkt] (P)  circle (1.875pt) node[left]  {\footnotesize $P$};
  \fill [color=colPkt] (A)  circle (1.875pt) node[above] {\footnotesize $A$};
  \fill [color=colPkt] (B)  circle (1.875pt) node[above] {\footnotesize $B$};
  \fill [color=colPkt] (C)  circle (1.875pt) node[below] {\footnotesize $C$};
  \fill [color=colPkt] (A') circle (1.875pt) node[above] {\footnotesize $A'$};
  \fill [color=colPkt] (B') circle (1.875pt) node[below] {\footnotesize $B'$};
  \fill [color=colPkt] (C') circle (1.875pt) node[below] {\footnotesize $C'$};
\end{tikzpicture}
      \caption{Theorem von Desargues}
    \end{figure}
  \end{thm}

  \begin{proof}
    fehlt noch...
  \end{proof}

  Ausgehend von diesem Satz können zwei wichtige Folgerungen gezogen werden. Nach dem Dualitätsprinzip existiert auch das duale
  Theorem von Desargues$^*$. Eine weitere wichtige Folgerung ist der Scheren-Satz.

  \begin{thm}[Scheren-Satz] \ \\
    Sind $ABCD$ und $A'B'C'D'$ Vierecke in einer projektiven Ebene, mit alternierenden Ecken auf verschiedenen Geraden und gelten
    die Beziehungen $AB\parallel A'B'$, $BC\parallel B'C'$, sowie $AD\parallel A'D'$, so gilt auch $CD \parallel C'D'$.

    \begin{figure}[ht]
      \begin{tikzpicture}[line cap=round,line join=round,>=triangle 45,scale=0.65]

  \draw (2,1) 	  coordinate (A);
  \draw (2.25,0) 	  coordinate (B);
  \draw (2.64,1.32) coordinate (C);
  \draw (3.01,0) 	  coordinate (D);
  \draw (5.34,2.67) coordinate (A');
  \draw (6,0) 	  coordinate (B');
  \draw (7.04,3.52) coordinate (C');
  \draw (8.04,0) 	  coordinate (D');
		
  \draw [domain=-0.5:8] plot(\x,{(-0-1*\x)/-2});
  \draw [domain=-0.5:8.5] plot(\x,{(-0-0*\x)/2.25});
	
  \draw (D) --(A) --(B) --(C);
  \draw (D')--(A')--(B')--(C');
				
  \draw [dash pattern= on 3pt off 3pt] (C) --(D);
  \draw [dash pattern= on 3pt off 3pt] (C')--(D');
	
  \fill [color=colPkt] (A) circle (2.3pt) node[above] {\footnotesize $A$};
  \fill [color=colPkt] (B) circle (2.3pt) node[below] {\footnotesize $B$};
  \fill [color=colPkt] (C) circle (2.3pt) node[above] {\footnotesize $C$};
  \fill [color=colPkt] (D) circle (2.3pt) node[below] {\footnotesize $D$};
				
  \fill [color=colPkt] (A') circle (2.3pt) node[above] {\footnotesize $A'$};
  \fill [color=colPkt] (B') circle (2.3pt) node[below] {\footnotesize $B'$};
  \fill [color=colPkt] (C') circle (2.3pt) node[above] {\footnotesize $C'$};
  \fill [color=colPkt] (D') circle (2.3pt) node[below] {\footnotesize $D'$};
\end{tikzpicture}
      \caption{Der Scheren-Satz}
    \end{figure}
  \end{thm}

  \begin{proof}
    Setze $E=(B \vee C) \wedge (A \vee D)$ und analog $E' = (B' \vee C') \wedge (A' \vee D')$, dann enstehen zwei Dreiecke $ABE$ und
    $A'B'E'$, die zueinander parallele Seiten besitzen. Damit liegen sie in Perspektive bzgl. der Geraden durch unendlich fernen
    Punkte. Weiterhin folgt damit
    \begin{align*}
      = & [\mbox{Desargues}^*]\Rightarrow \mbox{ $ABE$ und $A'B'E'$ in Perspektive bzgl.
        Schnittpunkt $P$ der beiden Geraden} \\
      \Rightarrow & CDE, C'D'E' \mbox{ in Perspektive bzgl. $P$} \\
      \Rightarrow & [\mbox{Desargues}] \Rightarrow CD \parallel C'D'
    \end{align*}
  \end{proof}

  \begin{defi}[Pappos-Ebene] \ \\
    Eine projektive Ebene, die das "`Axiom von Pappos"' erfüllt, heißt Pappos-Ebene.
  \end{defi}

  \begin{bem}
    Man kann zeigen, dass jede Pappos-Ebene isomorph ist zu einer Ebene $\mathbb{P}_2(\K)$.
  \end{bem}

\subsection*{Projektive Arithmetik}
    In diesem Abschnitt geht es um die projektive Addition und Multiplikation. Wir werden dafür den Begriff der Parallelität
    verwenden. Dies bedeutet jedoch lediglich, dass sich zwei Parallelen im "`Unendlichen"' schneiden.

    \begin{description}
        \item[Addition] Bevor wir die Addition begründen können, benötigen wir ein Grundgerüst. Dieses besteht aus $3$ Geraden,
              die sich nicht in einem gemeinsamen Punkt schneiden. Die eine Gerade ist die $x$-Achse, die andere die $y$-Achse und
              schließlich die Gerade unendlich ferner Punkte.
              \begin{enumerate}
                \item Konstruiere $g_1$ mit $g_1 \parallel x$-Achse
                \item Konstruiere $g_2 = a \vee $($g_1\cap y$-Achse)
                \item Konstruiere $g_3$ durch $b$ mit $g_3 \parallel y$-Achse
                \item Konstruiere $g_4$ durch $g_3 \wedge g_1$ mit $g_4 \parallel g_2$ \par
                      $\qquad \Rightarrow a+b := g_4 \wedge x$-Achse
              \end{enumerate}

              \begin{figure}[ht]
                \begin{tikzpicture}[line cap=round,line join=round,>=triangle 45,scale=0.75]

  \draw (2,0) coordinate (G);
  \draw (1.36,1.5) coordinate (H);
  \draw (3,0) coordinate (I);
  \draw (4.36,1.5) coordinate (J);
  \draw (5,0) coordinate (K);
	
  \draw [color=black!50!white] (H)--(G);
  \draw (I)--(J);
  \draw [color=black!50!white] (J)--(K);
  \draw [domain=-0.5:2.5,->] plot(\x,{(-0--2.96*\x)/2.68}) node[right]{$y$};
  \draw [domain=-0.5:6,->] plot(\x,{(-0-0*\x)/5}) node[right]{$x$};
  \draw [rotate around={0:(2.44,1)},dash pattern=on 3pt off 3pt] (2.44,1) ellipse (5.52cm and 2.99cm);
  \draw [domain=0.5:6] plot(\x,{(--7.49-0*\x)/5}) node[right] {$g_1$};
  \draw (4.75,3.78) node[right] {\footnotesize Gerade unendlich ferner Punkte};
		
  \foreach \x/\xtext in {0,2/a,3/b,5/a+b}
  \draw[shift={(\x,0)},color=black] (0pt,2pt) -- (0pt,-2pt) node[below] {\footnotesize $\xtext$};
\end{tikzpicture}
                \caption{projektive Addition}
              \end{figure}

              Wegen des Theorems von Desargues \ref{Theorem-von-Desargues} ist die Summe $a+b$ darüberhinaus unabhängig von der Wahl
              der Geraden $g_1$ ist.

              \begin{figure}[ht]
                \begin{tikzpicture}[line cap=round,line join=round,>=triangle 45,scale=0.75]

  \draw (2,0) 	   coordinate (G);
  \draw (1.36,1.5) coordinate (H);
  \draw (3,0) 	   coordinate (I);
  \draw (4.36,1.5) coordinate (J);
  \draw (5,0) 	   coordinate (K);
  \draw (2,2.2)    coordinate (M);
  \draw (5,2.2)    coordinate (N);

  \draw [fill=black!50!white] (H)--(G)--(M) --cycle;
  \draw (I)--(J);
  \draw [fill=black!50!white] (J)--(K)--(N) --cycle;
  \draw [domain=-0.5:2.5,-stealth] plot(\x,{(-0--2.96*\x)/2.68}) node[right]{\footnotesize $y$};
  \draw [domain=-0.5:6,-stealth] plot(\x,{(-0-0*\x)/5}) node[right]{\footnotesize $x$};
  \draw [domain=1.5:6] plot(\x,{(--11.02-0*\x)/5}) node[right] {\footnotesize $g'_1$};
  \draw [domain=0.76:6] plot(\x,{(--7.49-0*\x)/5}) node[right] {\footnotesize $g_1$};

  \foreach \x/\xtext in {0/,2/a,3/b,5/a+b}
  \draw[shift={(\x,0)},color=black] (0pt,2pt) -- (0pt,-2pt) node[below] {\footnotesize $\xtext$};
\end{tikzpicture}
                \caption{Unabhängigkeit der Summe $a+b$ von $g_1$}
              \end{figure}

        \item[Multiplikation] Für die projektive Multiplikation benötigen wir ein ähnliches Grundgerüst. Wir wählen wieder die $x$- und
              die $y$-Achse. Zusätlich sollen die Punkte $1_x$ und $1_y$ auf der $x$- bzw. auf der $y$-Achse liegen. Ausgehend von diesem
              Setup kann die Multiplikation nun wie folgt erklärt werden.

              \begin{enumerate}
                \item Konstruiere $g_1 = 1_x \vee 1_y$
                \item Konstruiere $g_2 = a \vee 1_y$
                \item Konstruiere $g_3$ durch $b$ mit $g_3 \parallel g_1$
                \item Konstruiere $g_4$ durch $g_3 \wedge y$-Achse mit $g_4 \parallel g_2$ \par
                      $\qquad \Rightarrow a\cdot b := g_4 \wedge x$-Achse
              \end{enumerate}

              \begin{figure}[ht]
                \begin{tikzpicture}[line cap=round,line join=round,>=triangle 45,x=1.0cm,y=1.0cm]
	
	\draw (0,0) coordinate (O);
	\draw (1.5,0) coordinate (X);
	\draw (3,0) coordinate (A);
	\draw (4,0) coordinate (B);
	\draw (8,0) coordinate (AB);
	\draw (1.25,0.75) coordinate (Y);
	\draw (3.33333,2) coordinate (F);
	
	\foreach \x/\xtext in {1.5/1_x,3/a,4/b,8/a\cdot b}
	\draw[shift={(\x,0)},color=black] (0pt,2pt) -- (0pt,-2pt) node[below] {\footnotesize $\xtext$};
	
	\draw [-stealth] (O) -- (8.5,0) node[right] {$x$};
	\draw [-stealth] (O) -- (4,2.4) node[right] {$y$};
	\draw (X) -- (Y);
	\draw [dash pattern=on 2pt off 2pt] (Y) -- (A);
	
	\draw (B) -- (F);
	\draw [dash pattern=on 2pt off 2pt] (F) -- (AB);
	
	\draw (Y) node[above] {\footnotesize $1_y$};
\end{tikzpicture}
                \caption{projektive Multiplikation}
              \end{figure}
    \end{description}

\subsection*{Zentral- und Parallelprojektion}

    \begin{defi}[projektiver $n$-dimensionaler Raum] \ \\
        $\displaystyle{\Pro_n(\K) = \Big( \K^{n+1} \backslash\lbrace 0 \rbrace \Big) \Big/_\sim }$ mit der Äquivalenzrelation
        $v \sim w: \; \Leftrightarrow \; v= \lambda \cdot w$ für ein $\lambda \in \K$ heißt projektiver $n$-dimensionaler
        Raum über einem Körper $\K$. \par
        $v\in \K^{n+1}$ bezeichnen die homogenen Koordinaten des projektiven Punktes $\langle v \rangle$.
    \end{defi}


  \textbf{Zentralprojektion} \par

  \begin{defi}[Abbildungsebene, Betrachtungspunkt, Zentralprojektion] \ \\
    Für die \textit{Abbildungsebene} (Projektionsebene) $\mathcal{A} = \big\{ x\in \K^n \mid \langle a,x \rangle + b = 0 \big\}$
    mit $a\in \K^{n}\backslash \{0\}, b\in \K$ und den \textit{Betrachtungspunkt} (Projektionszentrum)
    $z\in \K^n\backslash\{ \mathcal{A} \}$ ist die Zentralprojektion $Z_\pi: \K^n \rightarrow \mathcal{A}$ die Abbildung, die
    jedem Punkt $p\in \K^n$ den Schnittpunkt $p'$ der Geraden durch $p$ und $z$ mit $\mathcal{A}$ zuordnet,
    $Z_\pi(p) = g(z,p) \cap \mathcal{A}$, definiert für
    $p\notin \big\{ x\in \K^n \mid \langle x,a \rangle = \langle z,a \rangle \big\}$.
  \end{defi}

  \begin{figure}[ht]
    \tdplotsetmaincoords{70}{130}		% ---Betrachtung auf das Koordinatensystem---
\begin{tikzpicture}[tdplot_main_coords]
 \draw (4,0,0)  coordinate (P);
 \draw (-2,3,4) coordinate (W1);
 \draw (-2,5,4) coordinate (W2);
 \draw (-2,3,2) coordinate (W3);
 \draw (-2,5,2) coordinate (W4);
 \draw (-4,3,4) coordinate (W5);
 \draw (-4,5,4) coordinate (W6);
 \draw (-4,3,2) coordinate (W7);
 \draw (-4,5,2) coordinate (W8);		
			
 \draw (W1)--(W2)--(W4)--(W3) --cycle;
 \draw (W5)--(W6)--(W8)--(W7) --cycle;
 \draw (W1)--(W5);
 \draw (W2)--(W6);
 \draw (W3)--(W7);
 \draw (W4)--(W8);

 \filldraw [color=yellow, semitransparent](0,1.3,1) -- (0,1.3,3) -- (0,4,3) -- (0,4,1) -- cycle;
 \draw (0,4,1) node[right] {\footnotesize Abbildungsebene $\mathcal{A}$};
			
 \draw [color=black!50!white] (P)--(W1);
 \draw [color=black!50!white] (P)--(W2);
 \draw [color=black!50!white] (P)--(W3);
 \draw [color=black!50!white] (P)--(W4);
 %\draw (P)--(W5);
 %\draw (P)--(W6);
 %\draw (P)--(W7);
 %\draw (P)--(W8);
			
 \draw (0,2,2.6666667) -- (0,3.3333334,2.6666667) -- (0,3.3333334,1.3333334) -- (0,2,1.333334) -- cycle;
 \fill [color=colPkt] (0,2,2.6666667) circle (1pt);
 \fill [color=colPkt] (0,3.3333334,2.6666667) circle (1pt);
 \fill [color=colPkt] (0,3.3333334,1.3333334) circle (1pt);
 \fill [color=colPkt] (0,2,1.333334) circle (1pt);
			
 \fill [color=colPkt] (P)  circle (1.5pt) node[right = 0.5cm, color=black] {\footnotesize Betrachtungspunkt $z$};
 \fill [color=colPkt] (W1) circle (1.5pt);
 \fill [color=colPkt] (W2) circle (1.5pt);
 \fill [color=colPkt] (W3) circle (1.5pt);
 \fill [color=colPkt] (W4) circle (1.5pt);
 \fill [color=colPkt] (W5) circle (1.5pt);
 \fill [color=colPkt] (W6) circle (1.5pt);
 \fill [color=colPkt] (W7) circle (1.5pt);
 \fill [color=colPkt] (W8) circle (1.5pt);
\end{tikzpicture}
    \caption{Zentralprojektion}
  \end{figure}

  \begin{thm}
    Für die Zentralprojektion gilt:
    \begin{align*}
      Z_\pi(p) := p' = \underbrace{ \left( \frac{\langle a,p \rangle + b }{\langle a,p-z \rangle} \right) }_{ \lambda_1} \cdot z
                       \underbrace{-\left( \frac{\langle a,z \rangle + b }{\langle a,p-z \rangle} \right) }_{ \lambda_2} \cdot p
    \end{align*}
  \end{thm}

  \begin{bem}
  Durch diese Definition, lässt sich leicht überprüfen, dass $p'$ in der Abbildungsebene und auf der Geraden $g(z,p)$ liegt.
    \begin{enumerate}
      \item $p'\in g(p,z)$, da $\lambda_1 + \lambda_2 = 1$
      \item $p'\in \mathcal{A}$, da $\langle p',a \rangle +b = 0$
            \begin{align*}
              \Rightarrow \langle p',a \rangle & = \lambda_1 \cdot \langle z,a \rangle + \lambda_2 \cdot \langle p,a \rangle
                                                 = \frac{b\langle z,a \rangle - b\langle p,a \rangle}{\langle a,p-z \rangle}
                                                 = -b
            \end{align*}
    \end{enumerate}
  \end{bem}

  In homogenen Koordinaten: \par

  Sei $\mathcal{A} = \big\{ x \in \mathbb{P}_n(\K) \mid \langle \overline{a},x \rangle = 0 \big\}$ für $\overline{a}=\ve{a}{b} \in \K^{n+1}\backslash\{0\}$ die
  Abbildungsebene in homogenen Koordinaten. Die $(n-1)$-dimensionalen Ebenen in $\mathbb{P}_n(\K)$ werden durch projektive Punkte parametrisiert. Mit den homogenen
  Koordinaten $\overline{z}=\ve{z}{1}$ und $\overline{p}=\ve{p}{1}$ ergibt sich
  \begin{align*}
    Z_\pi(\overline{p}) & = \langle \overline{a},\overline{p} \rangle \cdot \overline{z} - \langle \overline{a},\overline{z} \rangle \cdot \overline{p} \\
                        & = \overline{z} \cdot \big( \overline{a}^T \overline{p} \big) - \big( \overline{a}^T \overline{z} \big) \cdot \overline{p} \\
                        & = \big( \overline{z}\overline{a}^T - \overline{a}^T\overline{z} \cdot \Id_{n+1} \big) \cdot \overline{p}
  \end{align*}


  \textbf{Parallelprojektion}

  \begin{defi}[Projektionsrichtung, Parallelprojektion] \ \\
    Sei $\mathcal{A}=\big\{ x \in \K^n \mid \langle a,x \rangle + b = 0 \big\}$ mit $a \in \K^n \backslash\{0\}$ und $b\in \K$ die
    Abbildungsebene (Projektionsebene) und    sei $v\in \K^n$ die \textit{Projektionsrichtung} mit $\langle a,v \rangle \neq 0$.
    Unter der Parallelprojektion $P_\pi: \K^n \rightarrow \mathcal{A}$ versteht man die Abbildung, die jedem Punkt $p\in \K^n$ den
    Schnittpunkt $p'$ der Geraden durch $p$ mit Richtung $v$ mit $\mathcal{A}$ zuordnet,
    $P_\pi = \big\{ x\in\K^n \mid x=p+\lambda \cdot v, \lambda \in \K \big\} \cap \mathcal{A}$.
  \end{defi}


  \begin{figure}[ht]
    \tdplotsetmaincoords{70}{130}		% ---Betrachtung auf das Koordinatensystem---
\begin{tikzpicture}[tdplot_main_coords]
  \draw (0,0,0) coordinate (W1);
  \draw (0,2,0) coordinate (W2);
  \draw (0,2,2) coordinate (W3);
  \draw (0,0,2) coordinate (W4);
  \draw (2,0,0) coordinate (W5);
  \draw (2,2,0) coordinate (W6);
  \draw (2,2,2) coordinate (W7);
  \draw (2,0,2) coordinate (W8);		
			
  \draw (W1)--(W2)--(W3)--(W4) --cycle;
  \draw (W5)--(W6)--(W7)--(W8) --cycle;
  \draw (W1)--(W5);
  \draw (W2)--(W6);
  \draw (W3)--(W7);
  \draw (W4)--(W8);
			
  \filldraw [color=yellow, semitransparent](-4,3,3) -- (-4,7,3) -- (-4,7,7) -- (-4,3,7) -- cycle;
  \draw (-4,4,6) node[above] {\footnotesize Abbildungsebene $\mathcal{A}$};
			
  \draw [->] (0,0,3) -- (-1,1,4);
  \draw [rotate=26] (-0.5,0.5,3.5) node[above] {\footnotesize Projektionsrichtung $\vec{v}$};
			
  \draw [color=black!50!white] (W1)--(-4,4,4);
  \draw [color=black!50!white] (W2)--(-4,6,4);
  \draw [color=black!50!white] (W3)--(-4,6,6);
  \draw [color=black!50!white] (W4)--(-4,4,6);
  \draw (-4,4,4)--(-4,6,4)--(-4,6,6)--(-4,4,6)--cycle;
			
  \fill [color=colPkt] (-4,4,4) circle (1pt);
  \fill [color=colPkt] (-4,6,4) circle (1pt);
  \fill [color=colPkt] (-4,6,6) circle (1pt);
  \fill [color=colPkt] (-4,4,6) circle (1pt);			
  \fill [color=colPkt] (W1) circle (1.5pt);
  \fill [color=colPkt] (W2) circle (1.5pt);
  \fill [color=colPkt] (W3) circle (1.5pt);
  \fill [color=colPkt] (W4) circle (1.5pt);
  \fill [color=colPkt] (W5) circle (1.5pt);
  \fill [color=colPkt] (W6) circle (1.5pt);
  \fill [color=colPkt] (W7) circle (1.5pt);
  \fill [color=colPkt] (W8) circle (1.5pt);
\end{tikzpicture}
    \caption{Parallelprojektion}
  \end{figure}


  \begin{thm}
    Für die Parallelprojektion gilt:
    \begin{align*}
      P_\pi(p) := p' = p - \underbrace{\left( \frac{\langle a,p \rangle + b}{\langle a,v \rangle} \right)}_{\lambda} \cdot v
    \end{align*}
  \end{thm}

  \begin{bem}
    Auch hier lassen sich zur Zentralprojektoin analoge Aussagen treffen. So ist
    \begin{enumerate}
      \item $p'\in \big\{ p + \lambda\cdot v \mid \lambda \in \K \big\}$
      \item $p'\in \mathcal{A}$, weil $\langle p',a \rangle + b = \langle p,a \rangle - \big( \langle a,p \rangle + b \big) + b = 0$
    \end{enumerate}
  \end{bem}

  In homogenen Koordinaten: \par

  Sei $\mathcal{A} = \big\{ x\in \mathbb{P}_n(\K) \mid \langle \overline{a},x \rangle = 0 \big\}$ für
  $\overline{a}= \ve{a}{b} \in \K^{n+1}\backslash\{0\}$ die Abbildungsebene in homogenen Koordinaten. Die
  homogenen Koordinaten der Punkte in $\{p+\lambda \cdot v \mid \lambda \in \K \}$ sind
  \begin{align*}
   \ve{p+\lambda \cdot v}{1} & = \ve{p}{1} + \lambda \cdot \ve{v}{0} = \overline{p} + \lambda \cdot \overline{v}.
    \intertext{Damit ergibt sich}
    P_\pi(\overline{p})
      & = \langle \overline{a},\overline{p} \rangle \cdot \overline{v} - \langle \overline{a},\overline{v} \rangle \cdot \overline{p} \\
      & = \overline{v} \cdot \big( \overline{a}^T \overline{p} \big) - \big( \overline{a}^T \overline{p} \big) \cdot \overline{p} \\
      & = \big( \overline{v}\overline{a}^T - \overline{a}^T \overline{v} \cdot \Id_{n+1} \big) \cdot \overline{p}
  \end{align*}

  Von der Vorstellung her ist besonders der $1$-dimensionale projektive Raum $\Pro_1(\K)$ über einem Körper $\K$ mit den homogenen Koordinaten
  $\ve{\alpha}{1},\ve{1}{0}$ bedeutend.

  \begin{figure}[ht]
    		\begin{tikzpicture}[line cap=round,line join=round,>=triangle 45,x=1.0cm,y=1.0cm,scale=2]
			%\draw [color=black!50!white,dash pattern=on 2pt off 2pt, xstep=1.0cm,ystep=1.0cm] (-0.5,-1.5) grid (3.5,1);
			
			\draw[-stealth,color=black] (-0.5,0) -- (3.5,0) node[right] {\footnotesize $x$};
			\foreach \x in {,1,2,3}
			\draw[shift={(\x,0)},color=black] (0pt,1pt) -- (0pt,-1pt) node[below right] {\footnotesize $\x$};
			
			\draw[-stealth,color=black] (0,1) -- (0,-1.5) node[left] {\footnotesize $y$};
			\foreach \y/\ytext in {-1/1,0/0}
			\draw[shift={(0,\y)},color=black] (1pt,0pt) -- (-1pt,0pt) node[below left] {\footnotesize $\ytext$};
			
			\clip(-0.5,-1.5) rectangle (3.5,1);
			\draw [color=orange, thick] (1,-1.5) -- (1,1) node[below right] {\footnotesize Abbildungsebene $x=1$};
			%\draw [dash pattern=on 2pt off 2pt,domain=-0.5:3.5] plot(\x,{(-1-0*\x)/1});

			\draw [dash pattern=on 2pt off 2pt] (1,-0.5) -- (2,-1);
			\draw [dash pattern=on 2pt off 2pt] (1,-0.33333) -- (3,-1);

			\fill [color=colPktKon] (1,0)     circle (0.75pt);
			\fill [color=colPkt] (1,-1)    circle (0.75pt);
			\fill [color=colPkt] (2,-1)    circle (0.75pt);
			\fill [color=colPkt] (0,0)     circle (0.75pt);
			\fill [color=colPktKon] (1,-0.5)  circle (0.75pt);
			\fill [color=colPkt] (3,-1)    circle (0.75pt);
			\fill [color=colPktKon] (1,-0.33) circle (0.75pt);
			\fill [color=colPkt] (0.68,0)  circle (0.75pt);	
			
			\draw [-stealth] (0,0)-- (1,-1);
			\draw [-stealth] (0,0)-- (1,-0.5);
			\draw [-stealth] (0,0)-- (1,-0.3333333);
			\draw [-stealth] (0,0)-- (1,0);

		\end{tikzpicture}
    \caption{Zentralprojektion im $1$-dim. projektiven Raum}
  \end{figure}

  \begin{align*}
    \ve{1}{1} \mapsto \ve{1}{1}, & \ve{2}{1} \mapsto \ve{1}{\frac{1}{2}}, \ve{3}{1} \mapsto \ve{1}{\frac{1}{3}} &
    \text{allgemein:} \ve{n}{1} \mapsto \ve{1}{\frac{1}{n}}
  \end{align*}

  Die drei (elementaren) Projektivitäten sind
  \begin{equation*}
    \begin{array}{rl}
      \text{Isometrie:}  & x \mapsto x+t \\
      \text{Streckung:}  & x \mapsto \alpha \cdot x \\
      \text{Projektion:} & x \mapsto \frac{1}{x}
    \end{array}
  \end{equation*}

  \begin{bem}
    Projektivitäten $\Pro_1(\R) \to \Pro_1(\R)$ werden durch Matrizen
    \begin{align*}
      M = \begin{pmatrix} a & b \\ c & d \end{pmatrix} \in GL_2(\R) \qquad (\mbox{d.h.} ad-bc \neq 0 )
    \end{align*}
    beschrieben:
    \begin{align*}
      \langle \ve{x}{1} \rangle \mapsto \langle M \ve{x}{1} \rangle
          = \begin{cases} \langle \ve{\frac{ax+b}{cx+d}}{1} \rangle, & \mbox{falls $cx+d \neq 0$} \\
                          \langle \ve{1}{0} \rangle,                 & \mbox{sonst.}
            \end{cases}
    \end{align*}
  \end{bem}

  \begin{bem} [Möbiustransformationen] \ \\
    \begin{enumerate}
      \item Für $\K=\C$ heißen die Abb. $\Pro_1(\K) \to \Pro_1(\K)$ mit $\displaystyle{ z\mapsto \frac{az+b}{cz+d}}$ auch Möbiustransformationen.
      \item $\displaystyle{\frac{ax+b}{cx+d}}$ nicht konstant $\Leftrightarrow$ $ad-bc \neq 0$
      \item Projektivitäten $\displaystyle{f(x) = \frac{ax+b}{cx+d}}$ werden von den elementaren Projektivitäten ($x\mapsto x+t, x\mapsto \alpha x, x\mapsto \frac{1}{x})$
            erzeugt. \par
            \begin{proof}
              Wegen $\displaystyle{\frac{ax+b}{cx+d}
              = \frac{ \frac{a}{c}\cdot (cx+d) + b - \frac{ad}{c} }{cx+d}} = \frac{a}{c} + \left( \frac{bc-ad}{c} \right) \cdot \left( \frac{1}{cx+d} \right)$ ergibt sich
              \begin{align*}
                \frac{ax+b}{cx+d}
                    = \begin{cases}
                          \frac{a}{c} + \left( \frac{bc-ad}{c} \right) \cdot \left( \frac{1}{cx+d} \right), & \mbox{falls } c\neq 0 \\
                          \frac{a}{d} \cdot x + \frac{b}{d}, & \mbox{sonst.}
                      \end{cases}
              \end{align*}
            \end{proof}
    \end{enumerate}
  \end{bem}

\subsection*{Doppelverhältnis}
  \begin{defi}[Doppelverhältnis] \ \\
    Für $4$ Punkte auf einer Geraden mit Koordinaten $p,q,r,s\in \Pro_1(\R)$ ist das \textit{Doppelverhältnis}
    \begin{equation*}
      \text{DV}(p,q;r,s) : = \left(\frac{r-p}{s-p}\right) \Big/ \left(\frac{r-q}{s-q}\right) = \frac{\text{TV}(r,s,p)}{\text{TV}(r,s,q)}
    \end{equation*}
  \end{defi}

  \begin{thm}
    Projektivitäten $\displaystyle{f(x)=\frac{ax+b}{cx+d}}$ erhalten das Doppelverhältnis.
  \end{thm}
  \begin{proof}
    Es genügt elementare Projektivitäten zu betrachten.\par
    \begin{enumerate}[(1)]
      \item klar für $x\mapsto x+t$ und $x\mapsto \alpha x$
      \item Betrachte $x\mapsto \frac{1}{x}$. Dann ergibt sich
            \begin{equation*}
                \text{DV}\left( \frac{1}{p}, \frac{1}{q}; \frac{1}{r}, \frac{1}{s} \right)
                  = \left( \frac{ \frac{1}{r}-\frac{1}{p} }{ \frac{1}{s}-\frac{1}{p} }\right)\Big/\left(\frac{ \frac{1}{r}-\frac{1}{q} }{ \frac{1}{s}-\frac{1}{q}}\right)
                  = \left( \frac{ \frac{p-r}{rp} }{ \frac{p-s}{sp} } \right) \Big/ \left( \frac{ \frac{q-r}{rq} }{ \frac{q-s}{sq} } \right)
                  = \text{DV}(p,q;r,s)
            \end{equation*}
    \end{enumerate}
  \end{proof}


  \textbf{Erweiterung zur oberen Halbebene.} Betrachte $\Pro_1(\R)$ als Rand der oberen komplexen Halbebene
  $\mathcal{H}=\lbrace z\in \C \mid \Imag(z) >0 \rbrace.$
  \begin{figure}[ht]
    \begin{tikzpicture}[line cap=round,line join=round,>=triangle 45,x=1.0cm,y=1.0cm]
	\draw [color=black!30!white,dash pattern=on 2pt off 2pt, xstep=1.0cm,ystep=1.0cm] (-3.5,0) grid (3,3.5);
	
	\draw [-stealth,color=black] (-3.5,0) -- (3,0);
	\draw [-stealth,color=black] (0,0) -- (0,3.5);

	\clip (-3.5,0) rectangle (3,3.5);
	\draw (0,0) circle (2cm);
	\draw (-3,0) -- (-3,3.5);
	\fill [color=black] (2,2) circle (1.5pt) node[above] {\footnotesize $z=x+iy$};

\end{tikzpicture}
    \caption{obere komplexe Halbebene}
  \end{figure}
  Erweitere $\displaystyle{f(x)=\frac{ax+b}{cx+d}}$ zu einer Abbildung $\mathcal{H}\to \mathcal{H}$:
  \begin{enumerate}[(1)]
    \item Die Abbildung $x\mapsto x+t$ lässt sich erweitern ($t\in \R$)
    \item Die Abbildung $x\mapsto \alpha \cdot x$ lässt sich erweitern für $\alpha > 0$ \par
          $\overline{z} \mapsto -\overline{z}$ ist eine Abb. von $\mathcal{H} \to \mathcal{H}$ und daher \par
          $ z\mapsto \alpha \overline{z}=-\lvert \alpha \rvert \overline{z}$ für $\alpha < 0$ ist eine Abb.
          von $\mathcal{H}\to \mathcal{H}$.
    \item Die Abbildung $x \mapsto \frac{1}{x}$ ist keine Abb. $\mathcal{H}\to \mathcal{H}$ (Inversion am Kreis) \par
          Aber: $z\mapsto \frac{1}{\overline{z}}$ ist eine Abbildung von $\mathcal{H}\to \mathcal{H}$ (Reflexion/Inversion am Einheitskreis)

  \end{enumerate}

  Damit ist die Erweiterung von $\displaystyle{f(x)=\frac{ax+b}{cx+d}}$ auf $\mathcal{H}\to \mathcal{H}$ eine Kombination von Erweiterungen elementaren Projektivitäten.
  Die Gruppe dieser speziellen Möbiustransformationen bezeichnen wir mit $\Sym(\mathcal{H})$.

  \begin{thm} \ \\
    Jede Abbildung aus $\Sym(\mathcal{H})$ bildet hyperbolische Geraden auf hyperbolische Geraden ab.
  \end{thm}
  \begin{proof}
   ....
  \end{proof}

  \textbf{Spiegelungen in $\mathcal{H}$}
  \begin{defi}
    Eine Abbildung $\mathcal{H}\to \mathcal{H}$ heißt hyperbolische Spiegelung, wenn sie eine Spiegelung an einer Achse parallel zur imaginären Achse, oder eine
    Inversion an einem Kreis mit Mittelpunkt auf der reellen Achse ist.
  \end{defi}

  \begin{bem}
    \begin{enumerate}[(1)]
      \item Die Abbildungen $z\mapsto -\overline{z}$ und $z\mapsto \frac{1}{\overline{z}}$ sind hyperbolische Spiegelungen
      \item Die Abbildungen $z\mapsto z+t$ und $z\mapsto \alpha z$ mit $\alpha>0$ sind Produkte von je zwei hyperbolischen Geraden
      \item Das Poincaré'sche Kreisscheibenmodell am Einheitskreis $\lbrace z\in \C \mid \lvert z \rvert <1 \rbrace$ wird durch
            \begin{equation*}
              z \mapsto \frac{1-zi}{z-i}
            \end{equation*}
            auf das Poincaré'sche Halbscheibenmodell abgebildet.
    \end{enumerate}
  \end{bem}
  \textbf{hyperbolische Distanzen}
  Die Möbiustransformationen bilden das Doppelverhältnis $\text{DV}(a,b;p,q)$ auf sich oder sein konjugiert komplexes ab.
  \begin{thm}
    Das Doppelverhältnis von $4$ Punkten auf einer hyperbolischen Gerade ist reell.
  \end{thm}
  \begin{proof}
    ....
  \end{proof}

  \begin{bem}
    Möbiustransformationen der oberen imaginären Achse bilden die "`Enden"' $\lbrace 0,\infty \rbrace$ auf sich ab. \par
    Für $a,b \in \R$ gilt
    \begin{align*}
      \text{DV}(ai,bi,0,\infty) & = \lim_{q\to \infty} \left(\frac{ai-0}{ai/q-1}\right) \Big/ \left(\frac{bi-0}{bi/q-1}\right) = \frac{a}{b} \\
      \text{DV}(ai,bi,\infty,0) & = \ldots =\frac{b}{a} \\
      \Rightarrow \left\vert \log{\frac{a}{b}} \right\vert & = \lvert \log a - \log b \rvert = \lvert \log b - \log a \rvert = \left\vert \log{\frac{b}{a}} \right\vert
    \end{align*}
  \end{bem}

  \begin{defi}
    Wir definieren $\displaystyle{ \left\vert \log \frac{a}{b} \right \vert}$ als die hyperbolische Distanz von $ai, bi \in \mathcal{H}$.
  \end{defi}

  \begin{bem}
    Für andere Punkte $A$,$B \in \mathcal{H}$ wird die Distanz über die hyperbolische Gerade $g=g(A,B)$ durch Möbiustransformationen ermittelt.
    \begin{equation*}
      F:\mathcal{H} \to \mathcal{H} \text{ mit } F(g) = \lbrace z\in \mathcal{H} \mid \Real(z) = 0 \rbrace: \dist(A,B) := \left\vert \frac{F(A)}{F(B)} \right\vert
    \end{equation*}
  \end{bem}
